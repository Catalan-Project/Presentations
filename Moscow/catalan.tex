\documentclass[mathserif]{beamer}

\usepackage[utf8]{inputenc}
\usepackage[T2A]{fontenc}
\usepackage[english,russian]{babel}

\usepackage{pgf,tikz}
\usetikzlibrary{arrows}

\usepackage{setspace}

\usetheme[height=7mm]{Rochester}

\beamertemplatenavigationsymbolsempty

\def\hr#1#2{\href{#1}{\textcolor{red}{#2}}}

\newenvironment{items}[1][\labelitemi]{\begin{list}{#1}{\setlength{\topsep}{0pt}\setlength{\partopsep}{0pt}\setlength{\parsep}{0pt}\setlength{\itemsep}{\parskip}}}{\end{list}}

\def\fr#1#2{\leavevmode\kern.1em\raise.76ex\hbox{\scalebox{.9}{#1}}\kern-.15em/\kern-.1em\lower.40ex\hbox{\scalebox{.9}{#2}}}

\definecolor{green}{rgb}{0,.59,.43}
\definecolor{red}{rgb}{.82,.15,.07}
\def\bulflag{{\setlength{\fboxsep}{0pt}\setlength{\fboxrule}{.05pt}%
\raisebox{-1ex}{\fbox{\textcolor{red}{\rule[0ex]{5ex}{1ex}}\kern-5ex\textcolor{green}{\rule[1ex]{5ex}{1ex}}\kern-5ex\textcolor{white}{\rule[2ex]{5ex}{1ex}}}}}}

\title{~ \\ \huge Числа Каталана}
\author{Алекс Иванов Цветанов, \\ Маргарита Велчева Стефанова, \\ Йоана Петрова Кичева \\
~ \\
{\small{}преподаватель: Ваня Данова}}
\institute{Софийская математическая гимназия — София, Болгария \,\bulflag}
\date{1-е мая 2016}

\begin{document}

\nonfrenchspacing

\frame[plain]{%
\centerline{\small{}\textcolor{blue}{Международный конкурс «Математика и проектирование»}}
\bigskip
\bigskip
\bigskip
\titlepage
}%=============================================

\def\moverel#1#2#3{\par\kern#1\hskip#2{#3}}

\frame[plain]{%=========================================================
\centerline{Что имеется общего между следующими объектами?}

\bigskip\bigskip\bigskip\bigskip

\centerline{\includegraphics[width=\textwidth,clip]{mix}}

\moverel{-11ex}{12.5ex}{\includegraphics[width=.15\textwidth,clip]{necklace}}
\moverel{-18.5ex}{0ex}{\texttt{2~3~1~6~5~4~7}}
\moverel{2ex}{31ex}{\rotatebox{-16}{$A \; A \; A + + \, A \; A + A + A \; A + + +$}}
\moverel{-25ex}{23ex}{\rotatebox{75}{\texttt{( ( ) ( ) ) ( ( ( ) ) ) ( )}}}

\bigskip\bigskip\bigskip

Этому вопросу посвящена данная работа.
}

\begin{frame}[fragile]%=========================================================
\frametitle{Введение}
В 1838\,г. математик Э.\,Ш.\,Каталан исследовал вопрос о числе способов расстановки скобок в выражениях.
Эти числа, для всевозможных длин $n$ выражения, сегодня называются числами Каталана.

\medskip

Впоследствии оказалось, что те же самые числа перечисляют и самые разные совокупности объектов, внешне не схожие между собой и заведомо отличающиеся от выражений.

\medskip

Назовем «каталановой» любую совокупность объектов, перечисляемую числами Каталана.

\medskip

Покажем некоторые из самых интересных таких совокупностей.

Для каждой совокупности свойство быть «каталановой» устанавливаем указанием взаимно-однозначного соответствия (биекцию) между ней и уже известной такой совокупностью.

\end{frame}

\begin{frame}[fragile]%=========================================================
\frametitle{Обособление подвыражений}

Рассмотрим задачу, схожую с рассмотренной Каталаном.

\medskip

Пусть дано выражение из $n$ операций сложения $n+1$ чисел.
Сколькими способами в нем возможно обособить подвыражения?

\medskip

Например, для $n=7$ и выражения
\[A + A + A + A + A + A + A + A\]
одно из возможных разбиений, указывая подвыражения подчеркиванием:
\[\underline{A + \underline{A + A}} + \underline{\underline{\underline{A + A} + A} + \underline{A + A}}\]
или, то же самое, скобками:
\[(A + (A + A)) + (((A + A) + A) + (A + A)).\]
\end{frame}

\begin{frame}[fragile]%=========================================================
\frametitle{Разбиение как дерево}
Каждому разбиению выражения биективно соответствует двоичное дерево, отражающее иерархию подвыражений.

Для нашего примера:

\begin{center}
\begin{tikzpicture}[line cap=round,line join=round,very thick,draw=black,>=latex,x=1cm,y=1cm]
\node (top) {$+$}
child {node {$+$}
             child {node {$A$}}
             child {node {$+$}
                         child {node {$A$}}
                         child {node {$A$}}}}
      child[missing]
      child[missing]
      child[missing]
      child {node {$+$}
             child {node {$+$}
                         child {node {$+$}
                                     child {node {$A$}}
                                     child {node {$A$}}}
                         child {node {$A$}}}
             child[missing]
             child {node {$+$}
                    child {node {$A$}}
                    child {node {$A$}}}};
\end{tikzpicture}
\end{center}
\end{frame}

\begin{frame}[fragile]%=========================================================
\frametitle{Суфиксная запись и последовательности из 1 и \texttt{-}$\!$1}
Суфиксный обход дерева (левое — правое — корень) дает суфиксную запись выражения, биективно соответствующую дереву и таким образом — данному разбиению.

Для нашего примера:
\[A \; A \; A + + \, A \; A + A + A \; A + + +\,.\]

Поставив 1 и \verb.-.$\!$1 соответственно на местах $A$ и $+$ в суфиксной записи, получим последовательность из $n{+}1$ единиц и $n$ минус единиц — для нашего примера
\[1 \;\; 1 \;\; 1 \; {-}\!1 \; {-}\!1 \;\; 1 \;\; 1 \;\; {-}\!1 \;\; 1 \; {-}\!1 \;\; 1 \;\; 1 \;\; {-}\!1 \; {-}\!1 \; {-}\!1\,,\]
у которой все частичные суммы ${>}\,0$, а сумма равна 1.
И наоборот, любой такой последовательности соответствует суфиксная запись некоторого разбиения исходного выражения.
\end{frame}

\begin{frame}[fragile]%=========================================================
\frametitle{Правильные последовательности из 1 и \texttt{-}$\!$1}
Назовем «правильными порядка $n$» последовательности из $n\,{+}\,1$ единиц и $n$ минус единиц с положительными частичными суммами.

\medskip

Если найти число правильных последовательностей, найдем и число разбиений выражений с $n$ сложениями.

\bigskip

Нетрудно доказать, что \emph{любая} последовательность из $n\,{+}\,1$ единиц и $n$ минус единиц можно, притом единственным образом, циклически сдвинуть так, что все частичные суммы у новой последовательности были ${>}\,0$ (сумма всех членов, конечно, всегда 1).
\end{frame}

\begin{frame}[fragile]%=========================================================
\frametitle{Значение $n$-го числа Каталана}
Число всех последовательностей из $n{+}1$ единиц и $n$ минус единиц — ${2n+1\choose n}$, поскольку для единиц можно выбрать любые $n$ из всех $2n{+}1$ возможных мест.
А поскольку, в силу вышесказанного, из всех последовательностей, приводимых одна в другую циклическим сдвигом, только одна правильная, для получения числа правильных последовательностей разделим ${2n+1\choose n}$ на число $2n{+}1$ возможных сдвигов.

\medskip

Число разбиений выражения, или число $C_n$ Каталана для $n{>}0$ тогда равно
\[C_n=\frac1{2n+1}{2n+1\choose n}=\frac{(2n)!}{n!(n+1)!}=\frac1{n+1}{2n\choose n}\,.\]
Положим еще $C_0\,{=}\,1$.

\medskip

Итак, установлено, что множество правильных строк из 1 и \verb.-.$\!$1 и множество разбиений выражения — каталановы.
\end{frame}

\begin{frame}[fragile]%=========================================================
\frametitle{Ожерелья}

Из сказанного следует и что множество ожерелий из $n$ белых \\
и $n\,{+}\,1$ черных бусин тоже является каталановым (сочтем все строки вида $((n{+}1) \times 1,\,n \times -1)$ циклическими, а числа в них заменим на бусины).

\medskip

Рассмотренному выше примеру соответствует ожерелье

\bigskip

\centerline{\includegraphics[width=.25\textwidth,clip]{necklace}}

\bigskip

и всего таких ожерелий $C_7=429$.
\end{frame}

\begin{frame}[fragile]%=========================================================
\frametitle{Пары скобок}
Если из любой правильной последовательности 1 и \verb.-.$\!$1 убрать начальную единицу и все оставшиеся члены 1 заменить на \verb.(., а \verb.-.$\!$1 на \verb.)., получится строка из $n$ пар правильно сгрупированных (сбалансированных) скобок.
Для рассмотренного выше примера
\begin{center}
\texttt{1 1 1 -\kern-.5ex1 -\kern-.5ex1 1 1 -\kern-.5ex1 1 -\kern-.5ex1 1 1 -\kern-.5ex1 -\kern-.5ex1 -\kern-.5ex1}
\end{center}
получаем
\begin{center}
\verb.( ( ) ) ( ( ) ( ) ( ( ) ) ).\,.
\end{center}

Обратным действием любая сбалансированная строка из $n$ пар скобок превращается в правильную последовательность 1 и \verb.-.$\!$1.

\bigskip

Найденное взаимно-однозначное соответствие показывает, что сбалансированные строки скобок являются каталановым множеством.
\end{frame}

\begin{frame}[fragile]%=========================================================
\frametitle{Горные хребты}
\centerline{\includegraphics[width=\textwidth,clip]{mountain-ridges}}

\bigskip
\medskip

Заменив в любой сбалансированной строке \verb.(. и \verb.). на наклонные линии \verb./. и \verb.\. и связав эти линии, получается фигура, известная как «горный хребет».
Характеристическое свойство такой фигуры: она не пересекает горизонталь.

\bigskip

На рисунке показаны все горные хребты для $n\,{=}\,4$.

\bigskip

Очевидно, что соответствие между строками скобок и хребтами взаимно-однозначно и значит множество горных хребтов — каталаново.
\end{frame}

\begin{frame}[fragile]%=========================================================
\frametitle{Поддиагональные маршруты и полиомино}
\centerline{\includegraphics[width=.9\textwidth,clip]{grid-walks}}

\bigskip

«Хребет» можно построить и на диагонали в клетчатой сетке, например снизу: вместо \verb./. и \verb.\. проводим \rule{1.9ex}{.15ex} и \verb.|..
Получаются маршруты, непересекающие диагональ.

\medskip

Каждый маршрут является и границей поддиагонального полиомино, занимающего правую нижнюю клетку или пустого.

\medskip

На рисунке показаны все такие маршруты для $n\,{=}\,4$.

\medskip

Ясно, что совокупность поддиагональных маршрутов, также как и совокупность полиомино, являются каталановыми.
\end{frame}

\begin{frame}[fragile]%=========================================================
\frametitle{Стопки бревен (или монет)}
\centerline{\includegraphics[width=\textwidth,clip]{log-stacks}}

\bigskip
\medskip

Сколько различных стопок бревен можно возвести на основе из $n$ плотно расположенных бревен?

\medskip

Построив «горный хребет» в клетчатой сетке, как показано ниже, понятно, что «стопка бревен» — не что иное, как множество клеток под хребтом.
Поэтому множество стопок — каталаново.

\medskip

Клетки же, или «бревна», можно считать и треугольными.

\bigskip

\centerline{\includegraphics[width=.95\textwidth,clip]{ridges-logs}}
\end{frame}

\begin{frame}[fragile]%=========================================================
\frametitle{Стековые перестановки}
Перестановку некоторой последовательности $a_1,a_2,\dots,a_n$ с участием стека производят следующим образом.
Начиная с пустого стека и пустого результата, на каждом шагу делают одно из двух, пока возможно:
\begin{items}[]
\item[{\bf{}а}.] либо берут из очереди очередное $a_i$ и кладут в стек,
\item[{\bf{}б}.] либо берут из стека элемент и добавляют к результату.
\end{items}

\medskip

Например, из последовательности \,\texttt{1~2~3~4~5~6~7} можно получить \texttt{2~3~1~6~5~4~7}, применив последовательность действий {\bf{}а~а~б~а~б~б~а~а~а~б~б~б~а~б}.

\medskip

Заменив в последовательности действий для перестановки буквы {\bf{}а} и {\bf{}б} на \verb.(. и \verb.)., получим сбалансированную строку из $n$ пар скобок, а любой сбалансированной строке соответствует стековая перестановка — имеет место биекция между этими двумя множествами объектов.

\medskip

Итак, стековые перестановки есть каталаново множество.
\end{frame}

\begin{frame}[fragile]%=========================================================
\frametitle{Упорядоченные леса и деревья: определения}
Корневое дерево (КД) — это пара (вершина,\,лес).
Точнее, это непустое корневое дерево.
Имеется ровно одно пустое КД.

\medskip

Лес — множество, возможно пустое, корневых деревьев.

\medskip

Когда множество в определении леса упорядочено, лес и соответственно деревья являются упорядоченными.
Именно такие деревья и леса рассматриваются дальше.

\medskip

Согласно определениям КД и леса, каждому лесу с $n\,{\ge}\,0$ вершинами однозначно соответствует КД с $n\,{+}\,1$ вершинами, и наоборот — каждому КД с $n\,{\ge}\,1$ вершинами однозначно соответствует лес с $n{-}1$ вершинами.
Таким образом, для каждого $n\,{\ge}\,0$ имеется биективное соответствие между множествами $n\,{+}\,1$-вершинных корневых деревьев и $n$-вершинных лесов.
\end{frame}

\begin{frame}[fragile]%=========================================================
\frametitle{Леса и строки скобок}
Сбалансированной строке из $n$ пар скобок поставим в соответствие лес по следующему правилу: \\
\begin{items}[]
\item \emph{соседним парам скобок соответствуют соседние (под)деревья, а вложенным скобкам — деревья, наследующие одно другое.}
\end{items}

\medskip

Например строке 

\centerline{\texttt{( ( ) ( ) ) ( ( ( ) ) ) ( )}}

соответствует лес

\centerline{\includegraphics[width=.20\textwidth,clip]{forest}\quad.}

\medskip

Соответствие биективно: любому лесу из $n$ вершин сопоставляется единственная строка из $n$ сбалансированных пар скобок.

\medskip

Поэтому лес является каталановым множеством.
\end{frame}

\begin{frame}[fragile]%=========================================================
\frametitle{Леса и двоичные деревья}
Двоичное дерево (ДД) — это дерево, которое либо пусто, либо состоит из корня и двух наследников, отличаемых как левый и правый, которые сами есть ДД.

\medskip

Любому лесу сопоставим двоичное дерево согласно правилу:
\emph{
\begin{items}[•]
\item корням деревьев леса соответствуют вершины, справа наследующие одна другую по порядку следования; также и с наследниками любой вершины;
\item первому из наследников любой вершины соответствует левый наследник данной вершины.
\end{items}
}

\medskip

Пример получения ДД из леса:

\bigskip

\centerline{\includegraphics[width=.7\textwidth,clip]{forest-to-binary}}

\end{frame}

\begin{frame}[fragile]%=========================================================
\frametitle{Леса и двоичные деревья — продолжение}
Соответствие имеет силу и в обратном направлении: любому ДД единственным образом сопоставляется некоторый лес.

\medskip

Этим установлено, что и множество двоичных деревьев является каталановым.

\medskip

Сверху на рисунке показаны все леса порядка 4, а внизу — соответствующие им двоичные деревья с 4 вершинами.

\bigskip
\medskip

\centerline{\includegraphics[width=1.15\textwidth,clip]{trees}}
\end{frame}

\begin{frame}[fragile]%=========================================================
\frametitle{Строго двоичные деревья}
Строго двоичное дерево (СДД) — двоичное дерево, каждая вершина которого имеет либо 0, либо 2 наследника.

\medskip

Нетрудно сообразить, что у каждого СДД имеется листьев на один больше, чем внутренних вершин.

\medskip

Деревья арифметических выражений с двуместными операциями, разбитых на подвыражения, которые рассматривали вначале — именно СДД.

\medskip

\begin{minipage}{.66\textwidth}
\small
Из непустого двоичного дерева можно получить СДД, добавив 1 или 2 наследника каждой вершине, у которой они 1 или 0.
Обратное преобразование — из СДД удалить все листья и соответствующие им ребра.
Этими двумя преобразованиями устанавливается биекция между двоичными деревьями с $n\,{\ge}\,1$ вершинами и СДД с $n{+}1$ листьями (и $n$ внутренними вершинами).

\end{minipage}\hfill%
\begin{minipage}{.30\textwidth}
\centerline{\includegraphics[width=\textwidth,clip]{binary-to-strict}}
\end{minipage}
\end{frame}

\begin{frame}[fragile]%=========================================================
\frametitle{Итог «каталановости» деревьев и лесов}
В силу сказанного, для каждого $n\,{\ge}\,0$, число
\begin{items}[$\circ$]
\item $n$-вершинных лесов,
\item $n{+}1$-вершинных корневых деревьев,
\item $n$-вершинных двоичных деревьев,
\item строго двоичных деревьев с $n{+}1$ листьями \\
      (или с $n$ внутренними вершинами)
\end{items}
одно и то же и является числом Каталана $C_n$.
\end{frame}

\begin{frame}[fragile]%=========================================================
\frametitle{Разрезание лестницы на прямоугольники}
Если из фигуры типа «лестница» вырезать прямоугольник с одним углом — угол лестницы, а другой — одна из ступеней, останется одна или две лестницы.
Если их разрезать дальше, всего получится $n$ прямоугольников, по числу ступеней.

\medskip

Каждому такому разрезанию соответствует двоичное дерево с корнем данный прямоугольник и наследниками — лестницы-
соседи с обеих сторон прямоугольника.
Также и каждому двоичному дереву соответствует разрезание.

\medskip

Из этого следует, что разрезания есть каталаново множество.

\medskip

На рисунке — все разрезания лестницы с 4 ступенями.

\bigskip

\centerline{\includegraphics[width=.9\textwidth,clip]{staircase-tilings}}
\end{frame}

\begin{frame}[fragile]%=========================================================
\frametitle{Триангуляция выпуклого многоугольника}
\begin{minipage}{.60\textwidth}
\small
Для любой триангуляции выпуклого $n$-угольника построим двойственный граф — с вершинами для треугольников и ребрами для соседства треугольников.
У этого графа $n{-}2$ вершин и $n{-}3$ ребер и если его «подвесить» за любую вершину, получается двоичное дерево.
(Можно наперед пометить какую-нибудь сторону многоугольника, чтобы выбрать примыкающий к ней треугольник, какой бы он не оказался для конкретной триангуляции.)
\end{minipage}\hfill%
\begin{minipage}{.35\textwidth}
\centerline{\includegraphics[width=\textwidth,clip]{triangulation-to-tree}}
\end{minipage}

\medskip

Вместе с тем, любому $n{-}2$-вершинному двоичному дереву соответствует ровно одна такая триангуляция.

\medskip

В силу этого соответствия, множество триангуляций является каталановым.
\end{frame}

\begin{frame}[fragile]%=========================================================
\frametitle{Триангуляции шестиугольника}
На рисунке — все триангуляции выпуклого шестиугольника.
(Их число — $C_{6-2}=C_4=14.$)

\bigskip\bigskip\bigskip

\centerline{\includegraphics[width=\textwidth,clip]{triangulations}}
\end{frame}

\begin{frame}[fragile]%=========================================================
\frametitle{Разрезание окружности хордами}
\begin{spacing}{.8}
$2\kern.1ex{}n$ точек на окружности соединены парами без пересечений.

\medskip

Пометим одну из точек.
Хорда, на которой точка лежит, объявим корнем двоичного дерева.
Поддеревьями будут выпуклые фигуры по обеим сторонам от хорды — пустые либо разрезанные хордами.
При наличии хорд одна из них выбирается вершиной поддерева и идет дальнейшее ветвление.
В конце процесса получим $n$-вершинное двоичное дерево.

\medskip

Обратив процесс, из каждого $n$-вершинного ДД получаем разрезание окружности $n$ хордами без пересечений.

\medskip

В силу биекции между разрезаниями и двоичными деревьями, первое множество является каталановым.

\medskip

На рисунке — все разрезания с 4 хордами.
\end{spacing}

\bigskip

\centerline{\includegraphics[width=.9\textwidth,clip]{chords}}
\end{frame}

\begin{frame}[fragile]%=========================================================
\frametitle{Треугольник Каталана}
Последовательностей бесконечно много и каждая из них бесконечна.
Каждая последовательность начинается вторым членом предыдущей, а каждое число является суммой верхнего и левого от него.
{\small\begin{verbatim}
        1   1   1   1   1   1   1   1 ...
            1   2   3   4   5   6   7 ...
                2   5   9  14  20  27 ...
                    5  14  28  48  75 ...
                       14  42  90 165 ...
                           42 132 297 ...
                              132 429 ...
                                  429 ...
                                      ...
\end{verbatim}}
Начальные элементы строк треугольника есть не что иное, как $C_n$ для $n=0,1,2,\dots$.
Вообще, $j$-й элемент в строке $i$ для $i,j=0,1,2,\dots$ равен числу последовательностей из 
$i$ членов \texttt{-}$\!$1 и $i{+}j$ членов 1 с неотрицательными частичными суммами.
\end{frame}

\begin{frame}[fragile]%=========================================================
\frametitle{Программное определение треугольника и чисел $C_n$}
Простота образования треугольника Каталана дает удобный способ запрограммировать порождение его содержания.

\medskip

В первом из следующих определений на языке Haskell порождается последовательность последовательностей, т.\,е.\ сам треугольник.
Второе — последовательность чисел Каталана.

\begin{verbatim}
    catrows = iterate (scanl1 (+) . tail) [1,1..]
    cats = map head catrows
\end{verbatim}

При помощи данных определений можно проводить например такие вычисления:

\begin{verbatim}
    cats !! i             -- i-е число Каталана
    take n cats           -- n первых чисел Каталана
    catrows !! i !! j     -- j-й член строки i
    take n (catrows !! i) -- n первых чисел строки i
\end{verbatim}
\end{frame}

\begin{frame}[fragile]%=========================================================
\frametitle{Задачи}
Доказать следующие рекуррентные равенства для $n\,{\ge}\,0$:

\smallskip

\begin{items}[•]
\item $C_{n+1}=\frac{4n+2}{n+2}\,C_n$\,;
\item $C_{n+1}=C_0C_n+C_1C_{n-1}+\dots+C_nC_0$\,.
\end{items}

\bigskip

Доказать «каталановость» следующих множеств:
\begin{items}[•]
\item последовательности натуральных чисел $1\,{\le}\,a_1\,{\le}\,a_2\,{\le}\,\dots\,{\le}\,a_n$,\, для которых \,$a_i\,{\le}\,i$;
\item последовательности целых чисел $a_1,\dots,a_n$, такие что $a_i$ равно количеству $j$, для которых $j\,{<}\,i$ и $a_j\,{\le}\,a_i$;
\item перестановки чисел $1,2,\dots,n$, у которых длины всех убывающих подпоследовательностей не больше 2.
\end{items}

\bigskip

Доказать указанное выше свойство чисел треугольника Каталана, вкл.\ то, что среди них — числа Каталана.
\end{frame}

\begin{frame}[fragile]%=========================================================
\frametitle{Программное порождение каталановых объектов}
В целях изучения каталановых множеств нами построено несколько компьютерных программ.

\medskip

Одна из них порождает все правильные скобочные строки (СС), притом с минимальной разницей: каждая строка отличается от предыдущей обменом скобок только в двух местах.
Другая программа переводит скобочную строку в соответствующее двоичное дерево (ДД).

\medskip

Дальше, имеются программы для переводов:

\smallskip

\begin{items}[—]
\item из СС в пермутацию со стеком;
\item из СС в «стопку бревен»;
\item из ДД в разрезание лестницы;
\item из ДД в триангуляцию многоугольника;
\item и др.
\end{items}

\medskip

Отдельный набор программ обеспечивает графическое представление деревьев, маршрутов, ожерелий, триангуляций и~пр.\ объектов.
\end{frame}

\begin{frame}[fragile]%=========================================================
\frametitle{Заключительные замечания}
\small
Числа Каталана являются предметом неослабевающего интереса в математике, в том числе в школе.
Об этом свидетельствует обилие литературы по теме — только в последние годы вышло несколько монографий — а также то, что статья о числах Каталана считается длиннейшей в энциклопедии OEIS (\verb:oeis.org:).

\bigskip

Исключительно разнообразие объектов, перечисление которых приводит к числам Каталана.

\bigskip

Во многих случаях изучение чисел Каталана и связанных с ними объектов не требует специфических математических знаний, а поэтому оно может проводиться и в школе.

\bigskip

Удачно «каталановость» множеств устанавливать путем нахождения взаимно-однозначного соответствия с уже знакомыми множествами, как мы и делали.
Это более полезно по сравнению со счетными методами ввиду того, что позволяет не только найти размер множества, но и получить представление о его строении.
Метод приведения также развивает способность к абстрактному мышлению: для изучения свойств одних объектов их заменяют другими.
\end{frame}

\begin{frame}[fragile]%=========================================================
 \frametitle{Заключительные замечания (продолжение)}
\small
Приведение одного объекта к другому часто имеет алгоритмический характер, с условиями и повторениями, что позволяет применять компьютерные программы для порождения и приведения самого разного вида объектов.
Если, например, составлен алгоритм для последовательного порождения объектов данного множества, он может применяться и для всех других множеств, к которым данное можно привести.

\bigskip

В большинстве случаев множествами, к которым удобно привести данный вид объектов, являются выражения со скобками и деревья, т.\,е.\ общие представления иерархии.
Тогда уместно ориентироваться на разработку алгоритмов порождения именно таких множеств.
\end{frame}

\begin{frame}[fragile]%=========================================================
\frametitle{Eugène Charles Catalan}
\begin{minipage}{.67\textwidth}
\small
Э.\,Каталан родился в г.\,Брюж (Бельгия), получил образование в Парижской политехнической школе и работал в области геометрии, те\-ории чисел, цепных дробей и комбинаторики.

\medskip

Рассмотренная им задача, в связи с которой появилось название «числа Каталана», была такой: сколькими способами можно расставить $n$ пар скобок в цепочке из $n\,{\ge}\,2$ букв так, чтобы внутри каждой пары было ровно два терма (буквы или выражения в скобках).

\medskip

В такой постановке нам сразу заметно прямое соответствие с двоичными деревьями, но в первой половине 19\,в.\ математическое понятие дерева было незнакомо.
\end{minipage}\hfill%
\begin{minipage}{.30\textwidth}
\includegraphics[height=.72\textheight,clip]{catalan-portrait}
\end{minipage}

\bigskip

До Каталана Л.\,Эйлер, решая задачу о триангуляции выпуклого многоугольника, фактически нашел последовательность чисел, названную именем Каталана.
\end{frame}

\begin{frame}[fragile]%=========================================================
\frametitle{Литература}
\tiny
\begin{thebibliography}{99}
\bibitem{Con}
Conway, J.\,H., Guy R.\,K. {\itshape The book of numbers}. Copernicus, 1996, pp.\,96-106.

\bibitem{Gri}
Grimaldi R.\,P. {\itshape Fibonacci and Catalan numbers: an introduction}, John Wiley \& Sons, 2012.

\bibitem{Kos}
Koshy Th., {\itshape Catalan numbers with applications}, Oxford University Press, 2009.

\bibitem{Rom}
Roman S., {\itshape An introduction to Catalan numbers}, Birkh\"{a}user, 2015.

\bibitem{Sta}
Stanley R.P., {\itshape Catalan numbers}, Cambridge University Press, 2015.

\bibitem{WkC}
{\itshape Catalan number}.
\texttt{http://en.wikipedia.org/wiki/Catalan\_number}.

\bibitem{WkT}
{\itshape Catalan's triangle}.
\texttt{http://en.wikipedia.org/wiki/Catalan\%27s\_triangle}.

\bibitem{MWC}
Stanley, R. and Weisstein, E\,.W. {\itshape Catalan Number. From MathWorld – A Wolfram Web Resource}.
\texttt{http://mathworld.wolfram.com/CatalanNumber.html}.

\bibitem{MWT}
Weisstein, E\,.W. Catalan's Triangle. {\itshape From MathWorld – A Wolfram Web Resource}.
\texttt{http://mathworld.wolfram.com/CatalansTriangle.html}.

\bibitem{SlC}
Sloane N.\,J.\,A. {\itshape Числа Каталана (последовательность A000108 в OEIS)}.
\texttt{http://oeis.org/A000108}.

\bibitem{SlT}
Sloane N.\,J.\,A. {\itshape Треугольник Каталана (последовательность A009766 в OEIS)}.
\texttt{http://oeis.org/A009766}.

\bibitem{MaF}
{\itshape Числа Каталана на The Math Forum}.
\texttt{http://mathforum.org/advanced/robertd/catalan.html}.

\bibitem{Par}
{\itshape Числовые треугольники, связанные с разбиениями}.
\texttt{http://en.wikiversity.org/wiki/Partition\_related\_number\_triangles}.
\end{thebibliography}
\end{frame}

\begin{frame}[fragile]%=========================================================
\frametitle{Программные технологии}
\def\fnmark#1{{\small}$^#1$}
\def\fnote#1#2{\hbox{\small\par\hskip-4ex$^#1\;$\texttt{#2}}}

\vglue3ex

В работе использованы:

\bigskip
\begin{items}[$\circ$]
\item язык и система \LaTeX\fnmark{1} с пакетом для изготовления презентаций \textsc{beamer}\fnmark{2}
\medskip
\item программа для создания чертежей \texttt{sp}\fnmark{3}
\medskip
\item языки программирования и описания рисунков \\
JavaScript\fnmark{4}, SVG\fnmark{5} и Haskell\fnmark{6}
\end{items}

\vglue8ex

\fnote{1}{http://ctan.org/pkg/latex}
\fnote{2}{http://ctan.org/pkg/beamer}
\fnote{3}{http://www.math.bas.bg/bantchev/sp}
\fnote{4}{http://ecma-international.org/publications/standards/Ecma-262.htm}
\fnote{5}{http://w3c.org/Graphics/SVG}
\fnote{6}{http://haskell.org}
\end{frame}

\begin{frame}[fragile]%=========================================================
\frametitle{Благодарности}
Выражаем благодарность \\
~ \\
\qquad Ване Дановой, \\
\qquad Бойко Банчеву и \\
\qquad Василу Тинчеву \\
~ \\
за оказанную помощь в работе по теме и изготовлении презентации.
\end{frame}

\frame[plain]{\centerline{\LARGE{}Спасибо за внимание!}}

\end{document}
